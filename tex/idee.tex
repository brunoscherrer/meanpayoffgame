\documentclass{article}
\usepackage{fullpage}
\usepackage[utf8x]{inputenc}
\usepackage{amssymb}
\usepackage{amsthm}
\usepackage{amsmath}
\usepackage{hyperref}
\usepackage{cleveref}
\usepackage{autonum}
\usepackage{dsfont}
\usepackage{stmaryrd}

\newtheorem{theorem}{Théorème}
\newtheorem{lemma}{Lemme}
\newtheorem{remark}{Remarque}
\newtheorem{corollary}{Corollaire}


\def\1{{\mathds 1}}
\def\G{{\cal G}}
\def\C{{\cal C}}
\def\V{{\cal V}}
\def\Ent{\mathbb N}
\def\R{\mathbb R}
\def\N{\mathfrak N}
\def\M{\mathfrak M}
\def\greed{\mathfrak G}
\newcommand{\suc}[1]{f(#1)}

\def\={\stackrel{def}{=}}
\newcommand{\intset}[1]{\llbracket #1 \rrbracket}
\newcommand{\intpart}[1]{\lceil #1 \rceil}

\begin{document}




\paragraph{Notations:}
\begin{align}
  v_{\mu,\nu} = (T_{\mu,\nu})^\infty
\end{align}

\paragraph{Algo PI standard:} on choisit $\mu_{k+1}$ tel que
\begin{align}
 T_{\mu_{k+1}} (T_{\mu_{k}})^\infty = T (T_{\mu_{k}})^\infty
\end{align}
Alors, la preuve de la croissance de PI est:
\begin{align}
  (T_{\mu_{k+1}})^\infty - (T_{\mu_k})^\infty & = \sum_{i=0}^{\infty} (T_{\mu_{k+1}})^{i+1} (T_{\mu_k})^\infty - (T_{\mu_{k+1}})^{i} (T_{\mu_k})^\infty \\
  & = \sum_{i=0}^{\infty} (T_{\mu_{k+1}})^{i} T (T_{\mu_k})^\infty - (T_{\mu_{k+1}})^{i} T_{\mu_k} (T_{\mu_k})^\infty \\
  & \ge 0
\end{align}
par monotonicité de $(T_{\mu_{k+1}})^{i}$.

La preuve de l'optimalité est: Supposons que $T_{\mu_{k+1}} (T_{\mu_{k}})^\infty = (T_{\mu_{k}})^\infty$. Alors,
\begin{align}
  (T_{\mu_{k}})^\infty = T (T_{\mu_{k}})^\infty.
\end{align}


\def\muc{\mu^c}
\def\nuc{\nu^c}

\paragraph{Idée de l'algo:}
On a $\mu$ politique stationnaire du joueur MAX, $\nu$ meilleur réponse à $\mu$:
\begin{align}
  (T_{\mu,\nu})^\infty = T_{\mu}^\infty.
\end{align}
On cherche un couple de politiques $\mu^m = (\mu_1,\mu_2,\dots,\mu_m)$ et $\nu^m = (\nu_1,\nu_2,\dots,\nu_m)$, un état $x$ et $c \le m$, tels que:
\begin{align}
  T_{\muc,\nuc}  \left[ T^{m-c} (T_{\mu})^\infty \right] & = T^c \left[ T^{m-c} (T_{\mu})^\infty \right], \\
  \1_x' P_{\muc,\nuc} & = \1_x',\\
  \1_x' T_{\muc,\nuc} \left[ T^{m-c} (T_{\mu})^\infty \right] & > \1_x' \left[ T^{m-c} (T_{\mu})^\infty \right].
\end{align}

\paragraph{Croissance}
On a:
\begin{align}
  (T_\mu)^\infty & = T_\mu (T_\mu)^\infty \\
  & \le T (T_\mu)^\infty \\
  & \dots \\
  &  \le T^{m-c} (T_\mu)^\infty \\
  & \le T^m (T_\mu)^\infty.
\end{align}
Alors,
\begin{align}
  ( T_{\muc,\nuc} )^\infty -  (T_\mu)^\infty &\ge  ( T_{\muc,\nuc} )^\infty - T^{m-c} (T_{\mu})^\infty \\
  & =  \sum_{i=0}^{\infty} ( T_{\muc,\nuc} )^{i+1} T^{m-c} (T_{\mu})^\infty - ( T_{\muc,\nuc} )^{i} T^{m-c} (T_{\mu})^\infty \\
  & = \sum_{i=0}^{\infty} ( T_{\muc,\nuc} )^{i} T^c T^{m-c} (T_{\mu})^\infty - ( T_{\muc,\nuc} )^{i} T^{m-c} (T_{\mu})^\infty \\
  & \ge 0
\end{align}
par monotonicité de $( T_{\muc,\nuc} )^{i}$.


\paragraph{Propriétés du cycle trouvé}


Soit $v$ une fonction quelconque. Pour tout $\bar\muc$ et $\bar\nuc$. 
\begin{align}
  %(T_{\bar\muc})^\infty - T^{m-c} (T_\mu)^\infty &=
  (T_{\bar\muc,\bar\nuc})^\infty - v %\\
  & =\sum_{i=0}^{\infty} (T_{\bar\muc,\bar\nuc})^{(i+1)} v - (T_{\bar\muc,\bar\nuc})^{i} v\\
  & = \sum_{i=0}^{\infty} (T_{\bar\muc,\bar\nuc})^{i} (T_{\bar\muc,\bar\nuc}) v - (T_{\bar\muc,\bar\nuc})^{i} v \\
  & = \sum_{i=0}^{\infty} (\gamma P_{\bar\muc,\bar\nuc})^{i}  ( (T_{\bar\muc,\bar\nuc}) v  - v).
\end{align}
Si $x$ est tel que $1_x' (P_{\bar\muc,\bar\nuc}) = 1_x'$, alors:
\begin{align}
  1_x' \left[ (T_{\bar\muc,\bar\nuc})^\infty - v \right] & = \frac{1_x'}{1-\gamma^c}  \left[ (T_{\bar\muc,\bar\nuc}) v - v \right], \\
\end{align}
c'est-à-dire:
\begin{align}
 1_x' (T_{\bar\muc,\bar\nuc})^\infty & = \frac{  1_x' [(T_{\bar\muc,\bar\nuc}) 0 ] }{1-\gamma^c}.
\end{align}

Notons:
\begin{align}
\M_c(x) & = \left\{ \muc ~;~ \1_x' P_{\muc} = 1'_x \right\}. \\
%\N_c(x) & = \left\{ \nuc ~;~ \1_x' \tilde P_{\nuc} = \1'_x \right\}
\end{align}

On a:
\begin{align}
1_x' \left[ \max_{\bar\muc \in \M_c(x)} \min_{\bar\nuc} (T_{\muc,\nuc})^\infty \right] = \frac{ 1_x' [ \max_{\bar\muc \in \M_c(x)} \min_{\bar\nuc} (T_{\bar\muc,\bar\nuc}) 0 ] }{1-\gamma^c}.
\end{align}

 

Si les deux joueurs trouvent ensemble un cycle, cela signifie que le joueur MAX peut imposer le cycle à MIN si MIN reste sur le cycle limite.
Sinon, MIN peut dégrader (réduire le bassin d'attraction du cycle) pour aller vers un autre cycle limite.
S'il n'y a pas d'autre cycle limite, MIN ne peut qu'accepter le cycle proposé par MAX (et dans ce cas, on avance d'un coup !)


\section{Algorithme en 2 phases}

Notations
\begin{align}
 \forall \mu \in \Pi_{max}^c,~ \nu_{*,c}(\mu) &= \arg\min_{\nu \in \tilde P^c} \left\{\nu ; v_\mu = v_{\mu,\nu} \right\}\\
  \C_{x,c} & = \{ \mu \in \Pi^c ~;~ \1_x' P_{\mu,\nu_{*,c}(\mu)} = \1_x' \},\\
 \forall v \in \R^n,~ \G_{x,c}(v) & = \{ \mu \in \C_{x,c} ~;~ T_\mu v = T^c v\},\\
  \V_{x,c} & = \left\{ v \in \R^n ~;~ \G_{x,c}(v) \in \C_{x,c} \right\}.
\end{align}

Propriété fondamentale
\begin{lemma}
  Pour tout $x,c$, l'ensemble
  $$
  \G_{x,c}(\V_{x,c})=\left\{ \G_{x,c}(v); v \in \V_{x,c} \right\}
$$
  contient au plus un élément.
\end{lemma}
Notons $v_{x,c}=v_{\G_{x,c}}(x)$.

Algorithme:

1) on calcule les $v_{x,c}$;

en résolvant un problème $c$-pas forcé à partir et à arriver en $x$.

il se peut que ce problème n'ait pas de solution possible (pas de chemin).


pour les noeuds max, on a une sur-estimation de la valeur $v_*$.

pour les noeuds min, on a une sous-estimation.


2) on en déduit en utilisant le fait que
$$
v_* = T^n w
$$
avec
\begin{align}
  w(x) & = O_{a,c} r(x,a) + \gamma v_{f(x,a),c}.
\end{align}

Preuve:  $T^n w \le v_*$. 

\bibliographystyle{plain}
\bibliography{biblio.bib} 

\end{document}
